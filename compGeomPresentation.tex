\documentclass{beamer}

\usepackage[resetfonts]{cmap}
\usepackage{lmodern}
\usepackage[english]{babel}
\usepackage{amsmath}
\usepackage{amsfonts}

%\AtBeginDocument{\shorthandoff{-"}} % Obejití chyby v podpoře češtiny v balíčku Babel.

\usepackage[latin2]{inputenc}
\usepackage[T1]{fontenc}
\usepackage[protrusion,expansion,step=1]{microtype}
\usepackage{hyperref}
\hypersetup{
  unicode=true,
  plainpages=false,
  pdfpagelabels=true,
  pdfpagemode=UseNone,
  pdfstartview=Fit,
  pdfpagelayout=SinglePage,
  pdfdisplaydoctitle=true,
  linkcolor={1 0 0},
  bookmarks=true,
  bookmarksopen=true,
  bookmarksopenlevel=1,
  bookmarksnumbered=true,
  pdfsubject={Image and Signal Processing, Convolution, Impulse Responses},
  pdfkeywords={Convolution, Dirac function}
}

%\usetheme{default}
%\usetheme{AnnArbor}
%\usetheme{Antibes}
%\usetheme{Bergen}
%\usetheme{Berkeley}
\usetheme{Berlin}
%\usetheme{Boadilla}
%\usetheme{CambridgeUS}
%\usetheme{Copenhagen}
%\usetheme{Darmstadt}
%\usetheme{Dresden}
%\usetheme{Frankfurt}
%\usetheme{Goettingen}
%\usetheme{Hannover}
%\usetheme{Ilmenau}
%\usetheme{JuanLesPins}
%\usetheme{Luebeck}
%\usetheme{Madrid}
%\usetheme{Malmoe}
%\usetheme{Marburg}
%\usetheme{Montpellier}
%\usetheme{PaloAlto}
%\usetheme{Pittsburgh}
%\usetheme{Rochester}
%\usetheme{Singapore}
%\usetheme{Szeged}
%\usetheme{Warsaw}

\usefonttheme{default}
%\usefonttheme{professionalfonts}
%\usefonttheme{serif}
%\usefonttheme{structurebold}
%\usefonttheme{structureitalicserif}
%\usefonttheme{structuresmallcapsserif}

%\usecolortheme{default}
%\usecolortheme{albatross}
\usecolortheme{beaver}
%\usecolortheme{beetle}
%\usecolortheme{crane}
%\usecolortheme{dolphin}
%\usecolortheme{dove}
%\usecolortheme{fly}
%\usecolortheme{lily}
%\usecolortheme{orchid}
%\usecolortheme{rose}
%\usecolortheme{seagull}
%\usecolortheme{seahorse}
%\usecolortheme{sidebartab}
%\usecolortheme{structure}
%\usecolortheme{whale}
%\usecolortheme{wolverine}

%\useinnertheme{default}
\useinnertheme{circles}
%\useinnertheme{inmargin}
%\useinnertheme{rectangles}
%\useinnertheme{rounded}

\useoutertheme{default}
%\useoutertheme{infolines}
%\useoutertheme{miniframes}
%\useoutertheme{shadow}
%\useoutertheme{sidebar}
%\useoutertheme{smoothbars}
%\useoutertheme{smoothtree}
%\useoutertheme{split}
%\useoutertheme{tree}
%\useoutertheme[subsection=false]{miniframes}

\setbeamercovered{transparent}
\setbeamertemplate{navigation symbols}{} % Odstranění navigační lišty.
%\setbeamertemplate{frametitle continuation}[from second]

\newcommand{\cssplit}[1]{\discretionary{#1}{#1}{#1}\kern 0pt} % Definice makra pro sazbu znaků opakovaných při zlomu na začátku dalsího řádku.
\newcommand{\spj}{\cssplit{-}} % Definice makra pro sazbu pri zlomu opakovaného  spojovníku.
\def\makro#1{\texttt{\char`\\#1}} % Makro pro sazbu názvů maker.
\let\bibautor\textsc % Vyznačení autora v seznamu literatury.
\let\bibnazev\textit % Vyznačení názvu díla v seznamu literatury.

\author{J\'an Bella \and Maxime Portaz}
\title{Delaunay Triangulation of Imprecise Points}
\subtitle{Preprocess and actually get a fast query time}
\institute{Grenoble INP}
\date{January 23, 2015} 

\setbeamerfont{footnote}{size=\tiny}

%\makeatletter
%\beamer@theme@subsectionfalse
%\makeatother
		
\begin{document}
\frame{\titlepage}

\begin{frame}
\frametitle{Outline}
\tableofcontents
\end{frame}

\section{Notions}
\frame{\tableofcontents[currentsection]}

\begin{frame}{Delaunay Triangulation}
there should be an image
\begin{figure}
		\centering
		\hspace*{-0.5cm}%\includegraphics[scale=0.35]{img/delaunay.png}
	\end{figure}

	
\end{frame}

\begin{frame}{Imprecise point}
\textbf{Imprecise point}: extending a point to some region
\end{frame}

\begin{frame}{Imprecise point}
\begin{itemize}
\item $|W| \rightarrow $ a set $W$
\item $|xy| \rightarrow $ distance between points x and y.
\item $DT_P \rightarrow $ Delaunay triangulation of set of points $P$
\item $NNP(v) \rightarrow $ nearest neighbour graph of $v \in P$ in $P \setminus \{v\}$
\item $d°G(v) \ rightarrow $ degree of point $v$ in graph $G$
\item $\dot{p} \rightarrow $ the center of imprecise point $p$
\item $\hat{p} \rightarrow $ an instance of imprecise point p.
\item $S, \dot{S}, \hat{S} $ analogously being the sets of these points
\item $D(p) \rightarrow $ the disk with center $p$ and radius $||\dot{p} NN\dot{S}(\dot{p})|| + 1$, in case of disjoint unit disks $S$
\item $W(q) = \{p \in S \setminus {q}; \hat{q} \in D(p)\}$ 
\end{itemize}

\end{frame}


\section{Algorithm}
\begin{frame}{Preprocessing}

\end{frame}

\begin{frame}

\end{frame}



\begin{frame}{Frame 2}
\begin{columns}
\column{.5\textwidth}
\begin{figure}
		\centering
		\hspace*{-0.5cm}%\includegraphics[scale=0.35]{img/****.png}
	\end{figure}
\column{.5\textwidth}

\end{columns}
\end{frame}

%\begin{frame}
%\centering
%\huge{Do you have any questions?}\\
%\end{frame}

\begin{frame}
\centering
\huge{Do you have any questions?}\\
\pause
\vspace*{1.5cm}
\centerline{\large{Thank you for your attention.}}
\end{frame}
\end{document}